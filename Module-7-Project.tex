% Options for packages loaded elsewhere
\PassOptionsToPackage{unicode}{hyperref}
\PassOptionsToPackage{hyphens}{url}
%
\documentclass[
]{article}
\usepackage{amsmath,amssymb}
\usepackage{lmodern}
\usepackage{iftex}
\ifPDFTeX
  \usepackage[T1]{fontenc}
  \usepackage[utf8]{inputenc}
  \usepackage{textcomp} % provide euro and other symbols
\else % if luatex or xetex
  \usepackage{unicode-math}
  \defaultfontfeatures{Scale=MatchLowercase}
  \defaultfontfeatures[\rmfamily]{Ligatures=TeX,Scale=1}
\fi
% Use upquote if available, for straight quotes in verbatim environments
\IfFileExists{upquote.sty}{\usepackage{upquote}}{}
\IfFileExists{microtype.sty}{% use microtype if available
  \usepackage[]{microtype}
  \UseMicrotypeSet[protrusion]{basicmath} % disable protrusion for tt fonts
}{}
\makeatletter
\@ifundefined{KOMAClassName}{% if non-KOMA class
  \IfFileExists{parskip.sty}{%
    \usepackage{parskip}
  }{% else
    \setlength{\parindent}{0pt}
    \setlength{\parskip}{6pt plus 2pt minus 1pt}}
}{% if KOMA class
  \KOMAoptions{parskip=half}}
\makeatother
\usepackage{xcolor}
\usepackage[margin=1in]{geometry}
\usepackage{color}
\usepackage{fancyvrb}
\newcommand{\VerbBar}{|}
\newcommand{\VERB}{\Verb[commandchars=\\\{\}]}
\DefineVerbatimEnvironment{Highlighting}{Verbatim}{commandchars=\\\{\}}
% Add ',fontsize=\small' for more characters per line
\usepackage{framed}
\definecolor{shadecolor}{RGB}{248,248,248}
\newenvironment{Shaded}{\begin{snugshade}}{\end{snugshade}}
\newcommand{\AlertTok}[1]{\textcolor[rgb]{0.94,0.16,0.16}{#1}}
\newcommand{\AnnotationTok}[1]{\textcolor[rgb]{0.56,0.35,0.01}{\textbf{\textit{#1}}}}
\newcommand{\AttributeTok}[1]{\textcolor[rgb]{0.77,0.63,0.00}{#1}}
\newcommand{\BaseNTok}[1]{\textcolor[rgb]{0.00,0.00,0.81}{#1}}
\newcommand{\BuiltInTok}[1]{#1}
\newcommand{\CharTok}[1]{\textcolor[rgb]{0.31,0.60,0.02}{#1}}
\newcommand{\CommentTok}[1]{\textcolor[rgb]{0.56,0.35,0.01}{\textit{#1}}}
\newcommand{\CommentVarTok}[1]{\textcolor[rgb]{0.56,0.35,0.01}{\textbf{\textit{#1}}}}
\newcommand{\ConstantTok}[1]{\textcolor[rgb]{0.00,0.00,0.00}{#1}}
\newcommand{\ControlFlowTok}[1]{\textcolor[rgb]{0.13,0.29,0.53}{\textbf{#1}}}
\newcommand{\DataTypeTok}[1]{\textcolor[rgb]{0.13,0.29,0.53}{#1}}
\newcommand{\DecValTok}[1]{\textcolor[rgb]{0.00,0.00,0.81}{#1}}
\newcommand{\DocumentationTok}[1]{\textcolor[rgb]{0.56,0.35,0.01}{\textbf{\textit{#1}}}}
\newcommand{\ErrorTok}[1]{\textcolor[rgb]{0.64,0.00,0.00}{\textbf{#1}}}
\newcommand{\ExtensionTok}[1]{#1}
\newcommand{\FloatTok}[1]{\textcolor[rgb]{0.00,0.00,0.81}{#1}}
\newcommand{\FunctionTok}[1]{\textcolor[rgb]{0.00,0.00,0.00}{#1}}
\newcommand{\ImportTok}[1]{#1}
\newcommand{\InformationTok}[1]{\textcolor[rgb]{0.56,0.35,0.01}{\textbf{\textit{#1}}}}
\newcommand{\KeywordTok}[1]{\textcolor[rgb]{0.13,0.29,0.53}{\textbf{#1}}}
\newcommand{\NormalTok}[1]{#1}
\newcommand{\OperatorTok}[1]{\textcolor[rgb]{0.81,0.36,0.00}{\textbf{#1}}}
\newcommand{\OtherTok}[1]{\textcolor[rgb]{0.56,0.35,0.01}{#1}}
\newcommand{\PreprocessorTok}[1]{\textcolor[rgb]{0.56,0.35,0.01}{\textit{#1}}}
\newcommand{\RegionMarkerTok}[1]{#1}
\newcommand{\SpecialCharTok}[1]{\textcolor[rgb]{0.00,0.00,0.00}{#1}}
\newcommand{\SpecialStringTok}[1]{\textcolor[rgb]{0.31,0.60,0.02}{#1}}
\newcommand{\StringTok}[1]{\textcolor[rgb]{0.31,0.60,0.02}{#1}}
\newcommand{\VariableTok}[1]{\textcolor[rgb]{0.00,0.00,0.00}{#1}}
\newcommand{\VerbatimStringTok}[1]{\textcolor[rgb]{0.31,0.60,0.02}{#1}}
\newcommand{\WarningTok}[1]{\textcolor[rgb]{0.56,0.35,0.01}{\textbf{\textit{#1}}}}
\usepackage{graphicx}
\makeatletter
\def\maxwidth{\ifdim\Gin@nat@width>\linewidth\linewidth\else\Gin@nat@width\fi}
\def\maxheight{\ifdim\Gin@nat@height>\textheight\textheight\else\Gin@nat@height\fi}
\makeatother
% Scale images if necessary, so that they will not overflow the page
% margins by default, and it is still possible to overwrite the defaults
% using explicit options in \includegraphics[width, height, ...]{}
\setkeys{Gin}{width=\maxwidth,height=\maxheight,keepaspectratio}
% Set default figure placement to htbp
\makeatletter
\def\fps@figure{htbp}
\makeatother
\setlength{\emergencystretch}{3em} % prevent overfull lines
\providecommand{\tightlist}{%
  \setlength{\itemsep}{0pt}\setlength{\parskip}{0pt}}
\setcounter{secnumdepth}{-\maxdimen} % remove section numbering
\newlength{\cslhangindent}
\setlength{\cslhangindent}{1.5em}
\newlength{\csllabelwidth}
\setlength{\csllabelwidth}{3em}
\newlength{\cslentryspacingunit} % times entry-spacing
\setlength{\cslentryspacingunit}{\parskip}
\newenvironment{CSLReferences}[2] % #1 hanging-ident, #2 entry spacing
 {% don't indent paragraphs
  \setlength{\parindent}{0pt}
  % turn on hanging indent if param 1 is 1
  \ifodd #1
  \let\oldpar\par
  \def\par{\hangindent=\cslhangindent\oldpar}
  \fi
  % set entry spacing
  \setlength{\parskip}{#2\cslentryspacingunit}
 }%
 {}
\usepackage{calc}
\newcommand{\CSLBlock}[1]{#1\hfill\break}
\newcommand{\CSLLeftMargin}[1]{\parbox[t]{\csllabelwidth}{#1}}
\newcommand{\CSLRightInline}[1]{\parbox[t]{\linewidth - \csllabelwidth}{#1}\break}
\newcommand{\CSLIndent}[1]{\hspace{\cslhangindent}#1}
\ifLuaTeX
  \usepackage{selnolig}  % disable illegal ligatures
\fi
\IfFileExists{bookmark.sty}{\usepackage{bookmark}}{\usepackage{hyperref}}
\IfFileExists{xurl.sty}{\usepackage{xurl}}{} % add URL line breaks if available
\urlstyle{same} % disable monospaced font for URLs
\hypersetup{
  pdftitle={Module 7 Project},
  pdfauthor={Callback Cats},
  hidelinks,
  pdfcreator={LaTeX via pandoc}}

\title{Module 7 Project}
\author{Callback Cats}
\date{2022-11-20}

\begin{document}
\maketitle

\hypertarget{introduction}{%
\section{Introduction}\label{introduction}}

The force produced by the contractions of skeletal muscle allows for
movement and stabilization of joints, among other functions. Muscle
generation of tension, or force, is based on the speed and length of the
muscle. In particular, sliding filament theory posits that sarcomeres
produce force by forming cross-bridges, protein complexes formed from
the binding of overlapping actin and myosin. This means that greater
cross-bridge formation leads to greater force production. It can then be
inferred that sarcomere maximal isometric force is determined by the
amount of actin-myosin overlap, which changes with muscle fiber length
and implies an optimal length for maximal force production (Fig. 1). In
this project, the force-length relationship is studied using human
forearm flexors, with muscle length manipulated by altering elbow angle.
We can then investigate the effect of muscle fiber length on force
generation, as well as how the FL relationship is impacted by fatigue.

\begin{figure}
\centering
\includegraphics{https://s3-us-west-2.amazonaws.com/courses-images/wp-content/uploads/sites/1940/2017/05/29212852/h-20tension-20relationship.png}
\caption{\textbf{Fig. 1.} The sarcomere force-length relationship. Graph
from Libretexts (2020).}
\end{figure}

A previous study (Sharma et al. (2021)), observed the FL relationship by
measuring maximum isometric contraction in the forearm over a range of
elbow angles, finding that maximal force production occurred at an
intermediate elbow angle of 90°. Muanjai et al. (2020) examined knee
extensor isometric maximal voluntary contractions (MVCs) at various
angles before and after eccentric exercise, finding a loss of force
after exercise. Though the authors attribute the force decrease to a
deficit in excitation-contraction coupling, it is clear that fatigue has
an effect on force production.

We seek to construct isometric force-angle curves for forearm flexor
MVCs at a number of elbow angles. The angle (θmax) at which non-fatigued
(control) maximum isometric force occurs will be compared to the θmax of
fatigued MVCs. The data collected here will help us to answer if
isometric MVCs from different subjects coalesce around a typical FL
relationship, and if θmax differs significantly between control and
fatigue FL relationships.

\hypertarget{methods}{%
\section{Methods}\label{methods}}

\hypertarget{setting-up-dataset-for-analysis}{%
\paragraph{Setting Up Dataset for
Analysis}\label{setting-up-dataset-for-analysis}}

\begin{Shaded}
\begin{Highlighting}[]
\FunctionTok{library}\NormalTok{(tidyverse)}
\FunctionTok{library}\NormalTok{(MuMIn)}
\FunctionTok{library}\NormalTok{(ggplot2)}
\FunctionTok{library}\NormalTok{(dplyr)}
\end{Highlighting}
\end{Shaded}

Here we are reading in the data and calculating the normalized force
data.

\begin{Shaded}
\begin{Highlighting}[]
\NormalTok{k }\OtherTok{\textless{}{-}} \FunctionTok{list.files}\NormalTok{(}\StringTok{"./Project 8 Data"}\NormalTok{, }\AttributeTok{full.names =}\NormalTok{ T,}\AttributeTok{pattern =} \StringTok{".csv"}\NormalTok{)}
\FunctionTok{print}\NormalTok{(k)}
\NormalTok{k.l }\OtherTok{\textless{}{-}} \FunctionTok{list}\NormalTok{()}

\ControlFlowTok{for}\NormalTok{(i }\ControlFlowTok{in}\NormalTok{ k)\{}
\NormalTok{  met.dat }\OtherTok{\textless{}{-}} \FunctionTok{unlist}\NormalTok{(}\FunctionTok{strsplit}\NormalTok{(i,}\StringTok{"\_"}\NormalTok{))}
\NormalTok{  sub }\OtherTok{\textless{}{-}}\NormalTok{ met.dat[}\DecValTok{2}\NormalTok{]}
\NormalTok{  ang }\OtherTok{\textless{}{-}} \FunctionTok{as.numeric}\NormalTok{(met.dat[}\DecValTok{3}\NormalTok{])}
\NormalTok{  exp }\OtherTok{\textless{}{-}} \FunctionTok{gsub}\NormalTok{(}\StringTok{"}\SpecialCharTok{\textbackslash{}\textbackslash{}}\StringTok{..+"}\NormalTok{,}\StringTok{""}\NormalTok{,met.dat[}\DecValTok{4}\NormalTok{])}
\NormalTok{  k.l[[i]] }\OtherTok{\textless{}{-}} \FunctionTok{read\_delim}\NormalTok{(i,}\AttributeTok{delim =} \StringTok{" "}\NormalTok{, }\AttributeTok{col\_names =} \FunctionTok{c}\NormalTok{(}\StringTok{"Reading"}\NormalTok{,}\StringTok{"Force"}\NormalTok{,}\StringTok{"Unit"}\NormalTok{), }\AttributeTok{id=}\StringTok{"Experiment"}\NormalTok{,}\AttributeTok{progress =} \ConstantTok{FALSE}\NormalTok{) }\SpecialCharTok{\%\textgreater{}\%}\FunctionTok{select}\NormalTok{(Force)}\SpecialCharTok{\%\textgreater{}\%}
    \FunctionTok{mutate}\NormalTok{(}\AttributeTok{sub=}\NormalTok{sub,}\AttributeTok{ang=}\NormalTok{ang,}\AttributeTok{exp=}\NormalTok{exp)}
\NormalTok{\}}

\NormalTok{data }\OtherTok{\textless{}{-}} \FunctionTok{do.call}\NormalTok{(rbind, k.l)}
\NormalTok{data }\OtherTok{\textless{}{-}}\NormalTok{ data}\SpecialCharTok{\%\textgreater{}\%}
  \FunctionTok{group\_by}\NormalTok{(sub,exp,ang)}\SpecialCharTok{\%\textgreater{}\%}
  \FunctionTok{summarise}\NormalTok{(}\AttributeTok{max.force=}\FunctionTok{max}\NormalTok{(}\FunctionTok{abs}\NormalTok{(Force), }\AttributeTok{na.rm=}\ConstantTok{TRUE}\NormalTok{), }\AttributeTok{n=}\FunctionTok{n}\NormalTok{())}

\NormalTok{data }\OtherTok{\textless{}{-}} \FunctionTok{na.omit}\NormalTok{(data)}

\NormalTok{data.joined }\OtherTok{\textless{}{-}}\NormalTok{ data}\SpecialCharTok{\%\textgreater{}\%}
  \FunctionTok{group\_by}\NormalTok{(sub,exp)}\SpecialCharTok{\%\textgreater{}\%}
  \FunctionTok{summarize}\NormalTok{(}\AttributeTok{max.f2 =} \FunctionTok{max}\NormalTok{(max.force))}\SpecialCharTok{\%\textgreater{}\%}
  \FunctionTok{left\_join}\NormalTok{(data)}\SpecialCharTok{\%\textgreater{}\%}
  \FunctionTok{mutate}\NormalTok{(}\AttributeTok{norm.force=}\NormalTok{max.force}\SpecialCharTok{/}\NormalTok{max.f2)}

\NormalTok{data.con }\OtherTok{\textless{}{-}} \FunctionTok{filter}\NormalTok{(data.joined,exp}\SpecialCharTok{==}\StringTok{"control"}\NormalTok{)}
\NormalTok{data.fat }\OtherTok{\textless{}{-}} \FunctionTok{filter}\NormalTok{(data.joined,exp}\SpecialCharTok{==}\StringTok{"fatigue"}\NormalTok{)}
\end{Highlighting}
\end{Shaded}

Here we are just separating the normalized force and angle values for
the control and fatigue data respectively.

\begin{Shaded}
\begin{Highlighting}[]
\NormalTok{ang.con }\OtherTok{\textless{}{-}}\NormalTok{ data.con}\SpecialCharTok{$}\NormalTok{ang}
\NormalTok{normF.con }\OtherTok{\textless{}{-}}\NormalTok{ data.con}\SpecialCharTok{$}\NormalTok{norm.force}

\NormalTok{ang.fat }\OtherTok{\textless{}{-}}\NormalTok{ data.fat}\SpecialCharTok{$}\NormalTok{ang}
\NormalTok{normF.fat }\OtherTok{\textless{}{-}}\NormalTok{ data.fat}\SpecialCharTok{$}\NormalTok{norm.force}
\end{Highlighting}
\end{Shaded}

\hypertarget{results}{%
\section{Results}\label{results}}

\hypertarget{control-condtions}{%
\subsection{Control Condtions}\label{control-condtions}}

\begin{Shaded}
\begin{Highlighting}[]
\NormalTok{poly.m2.con }\OtherTok{\textless{}{-}} \FunctionTok{lm}\NormalTok{(normF.con}\SpecialCharTok{\textasciitilde{}}\FunctionTok{poly}\NormalTok{(ang.con,}\DecValTok{2}\NormalTok{)) }\CommentTok{\#second order}
\NormalTok{poly.m3.con }\OtherTok{\textless{}{-}} \FunctionTok{lm}\NormalTok{(normF.con}\SpecialCharTok{\textasciitilde{}}\FunctionTok{poly}\NormalTok{(ang.con,}\DecValTok{3}\NormalTok{)) }\CommentTok{\#third order}
\NormalTok{poly.m4.con }\OtherTok{\textless{}{-}} \FunctionTok{lm}\NormalTok{(normF.con}\SpecialCharTok{\textasciitilde{}}\FunctionTok{poly}\NormalTok{(ang.con,}\DecValTok{4}\NormalTok{)) }\CommentTok{\#fourth order}

\FunctionTok{AICc}\NormalTok{(poly.m2.con,poly.m3.con,poly.m4.con)}
\end{Highlighting}
\end{Shaded}

\begin{verbatim}
##             df      AICc
## poly.m2.con  4 -1102.575
## poly.m3.con  5 -1100.887
## poly.m4.con  6 -1103.386
\end{verbatim}

\hypertarget{the-fourth-order-model-fits-the-best-to-data-because-it-is-has-the-lowest-aic-score.}{%
\paragraph{The fourth order model fits the best to data because it is
has the lowest AIC
score.}\label{the-fourth-order-model-fits-the-best-to-data-because-it-is-has-the-lowest-aic-score.}}

\begin{Shaded}
\begin{Highlighting}[]
\NormalTok{x.pred}\OtherTok{\textless{}{-}} \FunctionTok{seq}\NormalTok{(}\DecValTok{45}\NormalTok{,}\FloatTok{157.5}\NormalTok{,}\AttributeTok{length.out =} \DecValTok{1000}\NormalTok{) }\CommentTok{\#define 1000 angles from our range}

\NormalTok{normF.pred.con }\OtherTok{\textless{}{-}} \FunctionTok{predict}\NormalTok{(poly.m3.con,}\AttributeTok{newdata =} \FunctionTok{data.frame}\NormalTok{(}\AttributeTok{ang.con=}\NormalTok{x.pred)) }\CommentTok{\#predict the force using 1000 angles}
\end{Highlighting}
\end{Shaded}

\hypertarget{normative-force-vs.-angle-of-arm-for-control-conditions}{%
\paragraph{Normative Force vs.~Angle of Arm for Control
Conditions}\label{normative-force-vs.-angle-of-arm-for-control-conditions}}

\begin{Shaded}
\begin{Highlighting}[]
\FunctionTok{qplot}\NormalTok{(ang.con,normF.con)}\SpecialCharTok{+}\FunctionTok{geom\_point}\NormalTok{(}\FunctionTok{aes}\NormalTok{(}\AttributeTok{x=}\NormalTok{x.pred,}\AttributeTok{y=}\NormalTok{normF.pred.con),}\AttributeTok{col=}\StringTok{"red"}\NormalTok{)}\SpecialCharTok{+}\FunctionTok{geom\_point}\NormalTok{(}\FunctionTok{aes}\NormalTok{(}\AttributeTok{x=}\NormalTok{x.pred[}\FunctionTok{which.max}\NormalTok{(normF.pred.con)],}\AttributeTok{y=}\NormalTok{normF.pred.con[}\FunctionTok{which.max}\NormalTok{(normF.pred.con)]),}\AttributeTok{size=}\DecValTok{5}\NormalTok{,}\AttributeTok{col=}\StringTok{"blue"}\NormalTok{)}
\end{Highlighting}
\end{Shaded}

\includegraphics{Module-7-Project_files/figure-latex/con graphs-1.pdf}

\hypertarget{theta-max-for-control-conditions}{%
\paragraph{Theta Max for Control
Conditions}\label{theta-max-for-control-conditions}}

\begin{Shaded}
\begin{Highlighting}[]
\NormalTok{x.pred[}\FunctionTok{which.max}\NormalTok{(normF.pred.con)]  }
\end{Highlighting}
\end{Shaded}

\begin{verbatim}
## [1] 141.9595
\end{verbatim}

\hypertarget{fatigue-condtions}{%
\subsection{Fatigue Condtions}\label{fatigue-condtions}}

\begin{Shaded}
\begin{Highlighting}[]
\NormalTok{poly.m2.fat }\OtherTok{\textless{}{-}} \FunctionTok{lm}\NormalTok{(normF.fat}\SpecialCharTok{\textasciitilde{}}\FunctionTok{poly}\NormalTok{(ang.fat,}\DecValTok{2}\NormalTok{)) }\CommentTok{\#second order}
\NormalTok{poly.m3.fat }\OtherTok{\textless{}{-}} \FunctionTok{lm}\NormalTok{(normF.fat}\SpecialCharTok{\textasciitilde{}}\FunctionTok{poly}\NormalTok{(ang.fat,}\DecValTok{3}\NormalTok{)) }\CommentTok{\#third order}
\NormalTok{poly.m4.fat }\OtherTok{\textless{}{-}} \FunctionTok{lm}\NormalTok{(normF.fat}\SpecialCharTok{\textasciitilde{}}\FunctionTok{poly}\NormalTok{(ang.fat,}\DecValTok{4}\NormalTok{)) }\CommentTok{\#fourth order}

\FunctionTok{AICc}\NormalTok{(poly.m2.fat,poly.m3.fat,poly.m4.fat) }
\end{Highlighting}
\end{Shaded}

\begin{verbatim}
##             df      AICc
## poly.m2.fat  4 -919.2917
## poly.m3.fat  5 -922.6770
## poly.m4.fat  6 -921.3303
\end{verbatim}

\hypertarget{the-thirds-order-model-fits-to-the-data-the-best-because-it-is-has-the-lowest-aic-score.}{%
\paragraph{The thirds order model fits to the data the best because it
is has the lowest AIC
score.}\label{the-thirds-order-model-fits-to-the-data-the-best-because-it-is-has-the-lowest-aic-score.}}

\begin{Shaded}
\begin{Highlighting}[]
\NormalTok{normF.pred.fat }\OtherTok{\textless{}{-}} \FunctionTok{predict}\NormalTok{(poly.m4.fat,}\AttributeTok{newdata =} \FunctionTok{data.frame}\NormalTok{(}\AttributeTok{ang.fat=}\NormalTok{x.pred)) }\CommentTok{\#predict the force using 1000 angles}
\end{Highlighting}
\end{Shaded}

\hypertarget{normative-force-vs.-angle-of-arm-for-fatigue-conditions}{%
\paragraph{Normative Force vs.~Angle of Arm for Fatigue
Conditions}\label{normative-force-vs.-angle-of-arm-for-fatigue-conditions}}

\begin{Shaded}
\begin{Highlighting}[]
\FunctionTok{qplot}\NormalTok{(ang.fat,normF.fat)}\SpecialCharTok{+}\FunctionTok{geom\_point}\NormalTok{(}\FunctionTok{aes}\NormalTok{(}\AttributeTok{x=}\NormalTok{x.pred,}\AttributeTok{y=}\NormalTok{normF.pred.fat),}\AttributeTok{col=}\StringTok{"red"}\NormalTok{)}\SpecialCharTok{+}\FunctionTok{geom\_point}\NormalTok{(}\FunctionTok{aes}\NormalTok{(}\AttributeTok{x=}\NormalTok{x.pred[}\FunctionTok{which.max}\NormalTok{(normF.pred.fat)],}\AttributeTok{y=}\NormalTok{normF.pred.fat[}\FunctionTok{which.max}\NormalTok{(normF.pred.fat)]),}\AttributeTok{size=}\DecValTok{5}\NormalTok{,}\AttributeTok{col=}\StringTok{"blue"}\NormalTok{)}
\end{Highlighting}
\end{Shaded}

\includegraphics{Module-7-Project_files/figure-latex/Fat graph -1.pdf}

\begin{Shaded}
\begin{Highlighting}[]
\NormalTok{x.pred[}\FunctionTok{which.max}\NormalTok{(normF.pred.fat)]}\SpecialCharTok{{-}}\NormalTok{x.pred[}\FunctionTok{which.max}\NormalTok{(normF.pred.con)] }
\end{Highlighting}
\end{Shaded}

\begin{verbatim}
## [1] 15.54054
\end{verbatim}

\hypertarget{there-is-a-15.5-degree-shift-in-theta-max.}{%
\paragraph{There is a 15.5 degree shift in theta
max.}\label{there-is-a-15.5-degree-shift-in-theta-max.}}

\hypertarget{normative-force-vs.-angle-of-arm-grouped-by-control-and-fatigue-conditions}{%
\paragraph{Normative Force vs.~Angle of Arm Grouped by Control and
Fatigue
Conditions}\label{normative-force-vs.-angle-of-arm-grouped-by-control-and-fatigue-conditions}}

\begin{Shaded}
\begin{Highlighting}[]
\NormalTok{data.joined}\SpecialCharTok{\%\textgreater{}\%}
  \FunctionTok{ggplot}\NormalTok{(}\FunctionTok{aes}\NormalTok{(ang,norm.force,}\AttributeTok{col=}\NormalTok{exp))}\SpecialCharTok{+}\FunctionTok{geom\_point}\NormalTok{()}
\end{Highlighting}
\end{Shaded}

\includegraphics{Module-7-Project_files/figure-latex/overlaying graph-1.pdf}

Here we are calculating the AIC scores and fitting the models.

\begin{Shaded}
\begin{Highlighting}[]
\NormalTok{AICs }\OtherTok{\textless{}{-}}\NormalTok{ data.joined}\SpecialCharTok{\%\textgreater{}\%}
  \FunctionTok{group\_by}\NormalTok{(sub,exp)}\SpecialCharTok{\%\textgreater{}\%}
  \FunctionTok{summarize}\NormalTok{(}
    \AttributeTok{m2=}\FunctionTok{AICc}\NormalTok{(}\FunctionTok{lm}\NormalTok{(norm.force}\SpecialCharTok{\textasciitilde{}}\FunctionTok{poly}\NormalTok{(ang,}\DecValTok{2}\NormalTok{))), }\CommentTok{\#second order}
    \AttributeTok{m3=}\FunctionTok{AICc}\NormalTok{(}\FunctionTok{lm}\NormalTok{(norm.force}\SpecialCharTok{\textasciitilde{}}\FunctionTok{poly}\NormalTok{(ang,}\DecValTok{3}\NormalTok{))), }\CommentTok{\#third order}
    \AttributeTok{m4=}\FunctionTok{AICc}\NormalTok{(}\FunctionTok{lm}\NormalTok{(norm.force}\SpecialCharTok{\textasciitilde{}}\FunctionTok{poly}\NormalTok{(ang,}\DecValTok{4}\NormalTok{))) }\CommentTok{\#fourth order}
\NormalTok{  )}\SpecialCharTok{\%\textgreater{}\%}
  \FunctionTok{pivot\_longer}\NormalTok{(m2}\SpecialCharTok{:}\NormalTok{m4,}\AttributeTok{names\_to=}\StringTok{"model"}\NormalTok{,}\AttributeTok{values\_to=}\StringTok{"AICc"}\NormalTok{)}\SpecialCharTok{\%\textgreater{}\%}
  \FunctionTok{print}\NormalTok{()}
\end{Highlighting}
\end{Shaded}

\begin{verbatim}
## # A tibble: 165 x 4
## # Groups:   sub [28]
##    sub   exp     model    AICc
##    <chr> <chr>   <chr>   <dbl>
##  1 1     control m2    -130.  
##  2 1     control m3    -126.  
##  3 1     control m4    -117.  
##  4 1     fatigue m2     -87.0 
##  5 1     fatigue m3     -94.2 
##  6 1     fatigue m4     -85.6 
##  7 10    control m2     -70.1 
##  8 10    control m3     -52.1 
##  9 10    control m4       3.37
## 10 10    fatigue m2     -34.8 
## # ... with 155 more rows
\end{verbatim}

\begin{Shaded}
\begin{Highlighting}[]
\NormalTok{fits }\OtherTok{\textless{}{-}}\NormalTok{ data.joined}\SpecialCharTok{\%\textgreater{}\%}
  \FunctionTok{group\_by}\NormalTok{(sub,exp)}\SpecialCharTok{\%\textgreater{}\%}
  \FunctionTok{summarize}\NormalTok{(}
    \AttributeTok{m2=}\FunctionTok{predict}\NormalTok{(}\FunctionTok{lm}\NormalTok{(norm.force}\SpecialCharTok{\textasciitilde{}}\FunctionTok{poly}\NormalTok{(ang,}\DecValTok{2}\NormalTok{)),}\AttributeTok{newdata=}\FunctionTok{data.frame}\NormalTok{(}\AttributeTok{ang=}\NormalTok{x.pred)), }\CommentTok{\#second order}
    \AttributeTok{m3=}\FunctionTok{predict}\NormalTok{(}\FunctionTok{lm}\NormalTok{(norm.force}\SpecialCharTok{\textasciitilde{}}\FunctionTok{poly}\NormalTok{(ang,}\DecValTok{3}\NormalTok{)),}\AttributeTok{newdata=}\FunctionTok{data.frame}\NormalTok{(}\AttributeTok{ang=}\NormalTok{x.pred)), }\CommentTok{\#third order}
    \AttributeTok{m4=}\FunctionTok{predict}\NormalTok{(}\FunctionTok{lm}\NormalTok{(norm.force}\SpecialCharTok{\textasciitilde{}}\FunctionTok{poly}\NormalTok{(ang,}\DecValTok{4}\NormalTok{)),}\AttributeTok{newdata=}\FunctionTok{data.frame}\NormalTok{(}\AttributeTok{ang=}\NormalTok{x.pred)) }\CommentTok{\#fourth order}
\NormalTok{  )}\SpecialCharTok{\%\textgreater{}\%}
  \FunctionTok{pivot\_longer}\NormalTok{(m2}\SpecialCharTok{:}\NormalTok{m4,}\AttributeTok{names\_to=}\StringTok{"model"}\NormalTok{)}\SpecialCharTok{\%\textgreater{}\%}
  \FunctionTok{group\_by}\NormalTok{(sub,exp,model)}\SpecialCharTok{\%\textgreater{}\%}
  \FunctionTok{summarize}\NormalTok{(}\AttributeTok{theta\_max=}\NormalTok{x.pred[}\FunctionTok{which.max}\NormalTok{(value)])}\SpecialCharTok{\%\textgreater{}\%}
  \FunctionTok{print}\NormalTok{()}
\end{Highlighting}
\end{Shaded}

\begin{verbatim}
## # A tibble: 165 x 4
## # Groups:   sub, exp [55]
##    sub   exp     model theta_max
##    <chr> <chr>   <chr>     <dbl>
##  1 1     control m2         135.
##  2 1     control m3         135.
##  3 1     control m4         135.
##  4 1     fatigue m2         158.
##  5 1     fatigue m3         158.
##  6 1     fatigue m4         158.
##  7 10    control m2         136.
##  8 10    control m3         136.
##  9 10    control m4         136.
## 10 10    fatigue m2         158.
## # ... with 155 more rows
\end{verbatim}

\begin{Shaded}
\begin{Highlighting}[]
\NormalTok{best.models }\OtherTok{\textless{}{-}}\NormalTok{ fits}\SpecialCharTok{\%\textgreater{}\%}
  \FunctionTok{left\_join}\NormalTok{(AICs)}\SpecialCharTok{\%\textgreater{}\%}
  \FunctionTok{group\_by}\NormalTok{(sub,exp)}\SpecialCharTok{\%\textgreater{}\%}
  \FunctionTok{mutate}\NormalTok{(}\AttributeTok{best=}\NormalTok{AICc}\SpecialCharTok{==}\FunctionTok{min}\NormalTok{(AICc))}\SpecialCharTok{\%\textgreater{}\%}
  \FunctionTok{filter}\NormalTok{(best}\SpecialCharTok{==}\ConstantTok{TRUE}\NormalTok{)}\SpecialCharTok{\%\textgreater{}\%}
\NormalTok{  dplyr}\SpecialCharTok{::}\FunctionTok{select}\NormalTok{(}\SpecialCharTok{{-}}\NormalTok{best)}\SpecialCharTok{\%\textgreater{}\%}
  \FunctionTok{print}\NormalTok{()}
\end{Highlighting}
\end{Shaded}

\begin{verbatim}
## # A tibble: 55 x 5
## # Groups:   sub, exp [55]
##    sub   exp     model theta_max   AICc
##    <chr> <chr>   <chr>     <dbl>  <dbl>
##  1 1     control m2         135. -130. 
##  2 1     fatigue m3         158.  -94.2
##  3 10    control m2         136.  -70.1
##  4 10    fatigue m3         158.  -53.9
##  5 14    control m2         136. -115. 
##  6 14    fatigue m3         158.  -75.6
##  7 17    control m2         158.  -38.5
##  8 17    fatigue m3         158.  -38.4
##  9 18    control m2         158.  -27.9
## 10 18    fatigue m2         139.  -29.3
## # ... with 45 more rows
\end{verbatim}

Performing an ANOVA to investigate whether a shift of theta max is
different between the control and fatigue experiments.

\begin{Shaded}
\begin{Highlighting}[]
\FunctionTok{anova}\NormalTok{(}\FunctionTok{lm}\NormalTok{(theta\_max}\SpecialCharTok{\textasciitilde{}}\NormalTok{exp,best.models))}
\end{Highlighting}
\end{Shaded}

\begin{verbatim}
## Analysis of Variance Table
## 
## Response: theta_max
##           Df  Sum Sq Mean Sq F value  Pr(>F)  
## exp        1   754.3  754.27   3.415 0.07019 .
## Residuals 53 11706.1  220.87                  
## ---
## Signif. codes:  0 '***' 0.001 '**' 0.01 '*' 0.05 '.' 0.1 ' ' 1
\end{verbatim}

Calculating the mean shift with SEM.

\begin{Shaded}
\begin{Highlighting}[]
\NormalTok{best.models}\SpecialCharTok{\%\textgreater{}\%}
  \FunctionTok{pivot\_wider}\NormalTok{(}\AttributeTok{id\_cols=}\NormalTok{sub,}\AttributeTok{names\_from =}\NormalTok{ exp,}\AttributeTok{values\_from=}\NormalTok{theta\_max)}\SpecialCharTok{\%\textgreater{}\%}
  \FunctionTok{mutate}\NormalTok{(}\AttributeTok{shift=}\NormalTok{fatigue}\SpecialCharTok{{-}}\NormalTok{control)}\SpecialCharTok{\%\textgreater{}\%}
  \FunctionTok{ungroup}\NormalTok{()}\SpecialCharTok{\%\textgreater{}\%}
  \FunctionTok{summarize}\NormalTok{(}\AttributeTok{mean.shift=}\FunctionTok{abs}\NormalTok{(}\FunctionTok{mean}\NormalTok{(}\AttributeTok{na.rm=}\ConstantTok{TRUE}\NormalTok{,shift)),}\AttributeTok{se.shift=}\FunctionTok{sd}\NormalTok{(}\AttributeTok{na.rm=}\ConstantTok{TRUE}\NormalTok{,shift)}\SpecialCharTok{/}\FunctionTok{sqrt}\NormalTok{(}\FunctionTok{length}\NormalTok{(shift)))}
\end{Highlighting}
\end{Shaded}

\begin{verbatim}
## # A tibble: 1 x 2
##   mean.shift se.shift
##        <dbl>    <dbl>
## 1       8.02     4.33
\end{verbatim}

\hypertarget{there-is-a-8-degree-shift-in-the-mean-of-theta-max-between-the-control-and-fatigue-conditions.}{%
\paragraph{There is a 8 degree shift in the mean of theta max between
the control and fatigue
conditions.}\label{there-is-a-8-degree-shift-in-the-mean-of-theta-max-between-the-control-and-fatigue-conditions.}}

\hypertarget{discussion}{%
\section{Discussion}\label{discussion}}

\hypertarget{author-contributions}{%
\section{Author Contributions}\label{author-contributions}}

\hypertarget{references}{%
\section*{References}\label{references}}
\addcontentsline{toc}{section}{References}

\hypertarget{refs}{}
\begin{CSLReferences}{1}{0}
\leavevmode\vadjust pre{\hypertarget{ref-libretexts2020muscle}{}}%
Libretexts. 2020. {``9.3a: Force of Muscle Contraction.''}
\emph{Medicine LibreTexts}, August.
\url{https://med.libretexts.org/Bookshelves/Anatomy_and_Physiology/Book\%3A_Anatomy_and_Physiology_\%28Boundless\%29/9\%3A_Muscular_System/9.3\%3A_Control_of_Muscle_Tension/9.3A\%3A_Force_of_Muscle_Contraction}.

\leavevmode\vadjust pre{\hypertarget{ref-Muanjai2020knee}{}}%
Muanjai, Pornpimol, Mantas Mickevicius, Audrius Sniečkus, Saulė
Sipavičienė, Danguole Satkunskiene, Sigitas Kamandulis, and David A.
Jones. 2020. {``Low Frequency Fatigue and Changes in Muscle Fascicle
Length Following Eccentric Exercise of the Knee Extensors.''}
\emph{Experimental Physiology} 105 (3): 502--10.

\leavevmode\vadjust pre{\hypertarget{ref-Sharma2020angles}{}}%
Sharma, Hanjabam Barun, Arani Das, Prashant Tayade, and Kishore K.
Deepak. 2021. {``Recording of Length-Tension Relationship of Elbow
Flexors and Extensors by Varying Elbow Angle in Human.''} \emph{Indian
Journal of Physiology and Pharmacology}, January.
\url{https://ijpp.com/recording-of-length-tension-relationship-of-elbow-flexors-and-extensors-by-varying-elbow-angle-in-human/}.

\end{CSLReferences}

\end{document}
